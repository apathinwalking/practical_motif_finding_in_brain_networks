The \textit{order} of a graph is the number nodes it contains. The \textit{size} of a graph is the number of edges it contains. Two nodes are said to be \textit{connected} if an edge exists between them. The \textit{degree} of a node is the number of edges the node makes with other nodes.\\
A graph $H$ is a \textit{subgraph} of another graph $G$ if $V(H)$ is a subset of $V(G)$. A graph $J$ is an \textit{induced subgraph} of a graph $G$ if $J$ is a subgraph of $G$, and $J\neq G$.
A graph $G$ is \textit{isomorphic} to another graph $H$ if $G$ and $H$ have the same order, and are connected in the same way. That is, a bijection $b$ exists from $V(G)$ to $V(H)$ such that $uv\in E(G)$ if and only if $b(u)b(v)\in E(H)$. The bijection $b$ is called an $isomorphism$. An \textit{isomorphic class} of graphs is a collection of graphs which are all isomorphic to eachother.\\
An \textit{automorphism} of a graph is a shuffle of it's nodes such that an isomorphism exists from the original graph to it's shuffle. It should be noted that a concatenation of automorphisms is also an automorphism. A trivial automorphism exists as well - which is the automorphism in which no nodes move. An \textit{automorphism group} for a graph $G$ is the set of node shuffles (including the trivial shuffle) which produce a graph isomorphic to G.\\