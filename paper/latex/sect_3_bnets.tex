	Brain networks are characterized by different measures of brain connectivity. Brain connectivity is a broadly defined term for physical, statistically significant, or causal relationships between neural units or regions. There are three major types of brain networks based on 3 categories of connectivity. 
\subsection*{Connectivity}
	\textit{Structural brain networks} are the simplest of the three categories conceptually. These networks are based on \textit{structural connectivity} - which can refer to how neurons or neural regions are physically connected to one another. Networks of neurons and neuronal processes, traced in vitro, such as the neuronal network of the hermaphrodite Caenorhabditis elegans \cite{white86} are examples of structural brain networks. Regional networks might consist of a map of statistical relationships between morphological characteristics of the brain. These networks are found using imaging or tracing techniques which get a static picture of the brain such as DTI \cite{iturria07,iturria08} or structural MRI \cite{lerch06}.\\ 

	\textit{Functional connectivity}, the basis of \textit{functional brain networks}, consist of statistical relationships in time-series data of brain activity in anatomically distinct neural regions. Numerous functional brain networks have been assembled from both resting state \cite{van10} and task based \cite{rissman04,koshino05}fMRI data, PET \cite{friston93}, EEG \cite{lowe98, joudaki12}, and MEG \cite{stam09} scans.

	\textit{Effective connectivity} and \textit{effective brain networks} are an extension of functional connectivity. It differs from functional connectivity in that causality or direction is known between nodes in the network. This is akin to saying that effective brain networks are directed graphs. 

\subsection{Constructing Graphs from Brain Networks}
	%STRUCTURAL 
	Diffusion tensor imaging (DTI) is a method which allows for the assessment of the diffusion of water through open areas in the brain. When fibers are more organized, water diffuses through them with greater ease than when the fibers are randomly distributed. It is a measure of the restriction of the diffusion of water through tissue, which is measured by Fractional Anisotropy (FA) and diffusivity. Higher FA and lower diffusivity indicate a greater integrity of the tissue that is being looked at, which in turn gives a good estimate of whether an area of the brain is part of a tract. There are two approaches to DTI: the voxel-wise approach and tractography.\\
	In the voxel-wise approach you look at the whole brain, and look for places that are significantly different on FA or diffusivity. This tells you where there are differences in the brain. 
	In tractography you use predefined brain atlases to define the regions (tracts) in which to record. You then compare the levels of DA and diffusivity to discover which tracts have a greater connection. A brain network derived from DTI data would look different depending on which technique you used; thought both would be weighted structural networks. In the voxel-wise approach your networks nodes would be voxels - the smallest region of space a DTI can read. Edges would represent statistical relationships between voxels; how different they measure. As such this type of network would be an abstract one. It would not correlate exactly with physical connections in the brain. If the tractography approach was taken, nodes would be the predefined region in the brain atlas. The edges in the network would be a function of FA and diffusivity recorded between the two regions. 
	
	%FUNCTIONAL
	%fMRI
	When a region of your brain is active there is a delayed response of blood flow to this region. This is called the hemodynamic response and the blood oxygen level dependent (BOLD) signal is a measure of this. Changes occur in the magnetic structure of blood when it is depleted of oxygen. Thus an MRI scanner can detect changes in the activity of a brain region.
	A resting state fMRI evaluates connections and interactions between brain regions across time in an individual who is not assigned a specific task (at rest). Resting state fMRI also uses the BOLD signal. Often the BOLD signal is recorded at brain regions specified by brain atlas - based on anatomical parcellations of the brain. In doing this you can get a measure of the strength of functional connections between regions. Task-based or event-based fMRI measures the same in an individual who is performing a specified task. The BOLD signal is also measured in task-based fMRI. In task-based fMRI the BOLD signal may be contrasted against different tasks or points in a time-series. A brain network developed from either resting state or event-based fMRI would be a functional brain network. It would have brain regions from the brain atlas as nodes. Connection weights would be values calculated from the intensity of BOLD signals in conjunction with a measure of how close in time BOLD signals were recorded for adjacent nodes. 
	
	%EEG/MEG 
	EEG records electrical activity of the brain along the scalp. Generally it is a measure of electrical activity that is a function of neuronal activity. With a basic EEG you put electrodes on the scalp. There are different spectra (distributions of activity) which fluctuate over time. Different types of waves are categorized by their frequencies. Evoked potentials are derivatives of EEGs and they look at the activity of the brain immediately after an event (usually an outside stimulus). Different frequencies indicate a different level of activity in a brain region. MEG is another form of measurement of electrical activity in the brain. It is a technique for measuring the magnetic fields that result from electrical activity in the brain. It has a better spatial resolution than EEG but cannot reach deeper brain regions. A brain network derived from EEG or MEG data would be a functional brain network. Nodes in a graph representing this network would be brain regions which the EEG nodes are set to record. Edges would be a value representative of how close in time activation recorded by EEG is in one region is to another. 
