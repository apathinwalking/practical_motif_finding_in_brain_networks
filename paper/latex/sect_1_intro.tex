.
%%subsection::Network Motifs -- A summary of network motifs 
%%subsection::Purpose -- why did I do this SMP?
In the past 15 years there has been an explosion in the use of graph theoretical techniques in the study of brain networks\cite{bullmore09}. Graph theory, in a broad sense, is the study of distinct elements, \textit{vertices}, and the connections or relationships between them: \textit{edges}. Graph theory has been used in research in Although it is primarily used in math and computer science it has been used in a wide variety of disciplines including economics, sociology, biology, and physics. Graph theory, as a methodology, has been used to expose the computational underpinnings of the brain. Early research showed the ``small-world'' characteristics of the brain \cite{watts98}. This famous study helped put brain networks research on the map. Many other researchers investigated the small-world-ness of the brain networks of different populations, including people with alzheimer's disease \cite{starn07, sanz10, zhao12}, schizophrenia \cite{liou08} and ADHD \cite{wang09}. But thats not all that brain networks research has encompassed - a full review of graph theoretical techniques used in brain network research can be seen in \cite{sporns10}. 

\subsection{Network Motifs}
One graph theoretical measure of a network which has seen a lot of research in biology, but not as much in brain network research is the network motif. A network motif is a pattern found in a network whose occurence is statistically significant in comparison to other networks (usally random) with similar characteristics. Network motifs have been theorized to be computational building blocks of the networks they reside in. For example, in \cite{alon07} Alon describes 6 different types of feedforward loops found in protein interaction networks. In addition, in \cite{sporns04} Sporns describes the difference between a structural and functional motif, where a structural motif denotes the patterns of connections which form in a motif, and a functional motif, which denotes the subpatterns of effect or direction which can occur in a structural motif.\\
Currently available brain network data limits the search for functional motifs, however. Much of what brain network data is available, such as fMRI, EEG, MEG scans does not include any information about which direction information travels in the network (in graph theory terms, this is called an \textit{undirected network}). This is a limit to the technology; current brain scans do not have the temporal resolution to support such information. Although some methodology such as inferring granger causality or directly effecting brain regions with TMS can provide these results, they are not able to at the level which brain scanning technology can produce undirected networks. 
The difference is apparent in the research. Studies of motifs in brain networks have mostly looked at functional motifs \cite[p.107]{sporns10}. As such the research is limited in quantity and in scale. This has been stagnant even with the release of larger and larger datasets such as those available in the human connectome project \cite{biswal2010} and better technology for network motif detection (described in section 4).\\
\subsection{Purpose}
The purpose of the current study was to analyze and improve upon the methodological limits to network motif discovery in brain networks. First, technical details defining brain networks, and their limits are explored. Next, a review of network motif discovery is done. Improvements upon the current best known algorithm are explored, and a new approach to methodology and an algorithm are proposed. Following this a study using the proposed methodology and brain networks of 520 children with and without ADHD is done as proof of concept for the methodology. Finally, the methodology and results are analyzed and the current state of network motif discovery in brain networks is further reviewed. 