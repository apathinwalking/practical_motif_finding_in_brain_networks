\subsection{Results of Subgraph Analysis}
A significant difference was found in the expression of 3 different motifs in the resting state fMRI connectivity of individuals with ADHD against typically developing individuals. Comparable thresholds on similar datasets were not able to be found in the research. However, the greater frequency of these subgraphs in particular might be reflective of research suggesting that brain networks of individuals with ADHD lie more on the ordered side of the small-world network spectrum than random \cite{wang09}. This is further supported by the fact that both the present study and cited study look at resting state fMRI connectivity matrices. Further research and analysis must be done to be conclusive, however.\\
Additionally, it is apparent that 4-cycles are statistically under-represented across all thresholds of data, making the 4-cycle an antimotif. In an intuitive sense, this may be because the 4-cycle lies between more sparse structures such as 4-1 and dense structures like 4-4 or 4-5. Its lack of occurence may also be an effect of the small-world character of brain networks, where random connections are likely to disrupt the perfect order of the 4-cycle.\\
Further analysis of the data may reveal other significant patterns between thresholds. However, the current author is not equipped with a knowledge of how to perform this analysis correctly. 
\subsection{Future Plans}
\subsubsection{Comparison of Graph Isomorphism Algorithms}
In the future, the GTrieScanner source code could be modified to output the specific graphs that were sent into the nauty algorithm for automorphism queries, across a number of different networks. This way, we know exactly the types of graphs that the automorphism testing in the G-Tries algorithm must be optimized for. Each set of graphs would be held seperately, so that the computational time for each network with a different algorithm can be tested. We could pass each of sets of graphs into another algorithm such as \emph{Bliss} and we would record the total running time on the set. The total running time would be compared across the different networks. 
\subsubsection{Corrections to ReG-Tries algorithm}
The Re-GTries algorithm was not successful due to a conceptual mistake made by the author. However, the concept of the algorithm still may be salvageable. The ReG-Tries algorithm would have been viable if one was able to predict which subgraphs a motif decomposes into when it loses an edge. It is possible to predict the subgraph which will occur when removing a \textit{node} from our motif via the G-Trie, which gives us a sequence of subgraphs our motif is composed of. The decomposition concept of The ReG-Tries algorithm thus might still be useful in areas of research like single-node motifs \cite{echtermeyer11} or neuronal death - like that experienced in Alzheimer's disease. Another approach would be to reorganize the way in which the G-Trie is organized to allow us to have deterministic knowledge of the edges present in a motif. For example, we can know that if we currently have a cycle, if we remove any edge from that cycle we know that we will have a path. As well, for any k-connected motif, removal of exactly one edge will always produce graphs in the same isomorphic class. Is there a way that we could optimize the G-Trie to build on these types of graphs, which give us definite knowledge of the edge structure of the graph? And would this pseudo-G-Trie be able to perform anywhere near the speed at which the G-Tries algorithm performs, or be able to perform at all? 
\subsubsection{Further analysis of Motifs on the ADHD200 set}
It would be interesting to see if the overabundance of certain order 4 subgraphs in the brain networks of children with ADHD would present themselves as higher frequencies of related 5 order subgraphs. The calculation of all measures on the 520 datasets took a total of 2 hours of computation. Given the results reported here of GTrieScanner on the C. Elegans neuronal network, 5 order subgraphs also seem viable. 